% Created 2016-11-20 Sun 18:02
% Intended LaTeX compiler: pdflatex
%!TEX program = xelatex
\documentclass[12pt,spanish]{beamer}
\setbeamertemplate{navigation symbols}{}
\usecolortheme[RGB={7,29,66}]{structure}
\usepackage{tkz-euclide}
\usetkzobj{all}
\usepackage{pgf,tikz}
\usetikzlibrary{mindmap,trees,arrows}
\usepackage{tcolorbox}
\usepackage{fourier}
\usepackage{graphicx}
\usepackage{amssymb,amsmath}
\usepackage{polyglossia}
\setdefaultlanguage{spanish}
\usepackage[style=spanish]{csquotes}
\usepackage{siunitx}
\usepackage{xcolor}
\usepackage{booktabs}
\setbeamertemplate{caption}[numbered]
\usepackage{hyperref}
\definecolor{links}{HTML}{000000}
\setromanfont[Mapping={tex-text},Numbers={OldStyle},Ligatures=Rare]{LinLibertine}
\setsansfont[Mapping={tex-text},Numbers={OldStyle},Ligatures=Rare]{LinBiolinum}
\setmonofont[Mapping={tex-text},Numbers={OldStyle},Ligatures=Rare,Scale=0.8]{PragmataProMono}
\graphicspath{{/Users/HS/Dropbox/images/}}
\usebackgroundtemplate{\includegraphics[width=\paperwidth,height=\paperheight]{uacpp.pdf}}
\everymath{\displaystyle}
\def\NN{\mathbb{N}}
\def\RR{\mathbb{R}}
\def\ZZ{\mathbb{Z}}
\def\QQ{\mathbb{Q}}
\def\II{\mathbb{I}}
\definecolor{bluac}{RGB}{7,29,66}
\newcommand{\framedhref}[2]{\href{#1}{\fcolorbox{bluac}{bluac}{\textcolor{white}{#2}}}}
\newtheorem{teorema}{Teorema}[section]
\newtheorem{lema}[teorema]{Lema}
\newtheorem{proposicion}[teorema]{Proposición}
\newtheorem{corolario}[teorema]{Corolario}
\newtheorem{definicion}[teorema]{Definición}
\newtheorem{ejemplo}[teorema]{Ejemplo}
\newtheorem{nota}[teorema]{Nota}
\AtBeginSection[]{%
  \begin{frame}<beamer>
    \frametitle{Agenda}   \tableofcontents[sectionstyle=show/hide,subsectionstyle=hide/show/hide,currentsection]
  \end{frame}
  \addtocounter{framenumber}{-1}
}
\usetheme{Hytex}
\author{Hans Sigrist}
\date{\today}
\title{Episodios académicos que requieren procesos comunicativos}
\date{}
\institute[UAC]{UAC}
\hypersetup{
 pdfauthor={Hans Sigrist},
 pdftitle={Episodios académicos que requieren procesos comunicativos},
 pdfkeywords={ejercicio practico},
 pdfsubject={Episodios que requieren procesos comunicativos},
 pdfcreator={Emacs 25.1.1 (Org mode 9.0)}, 
 pdflang={English}}
\begin{document}

\maketitle


\begin{frame}[label={sec:org85c7890}]{Episodios académicos que requieren de procesos comunicativos}
Durante el primer semestre recién pasado, tuve oportunidad de observar diversos procesos que requieren de nuestra atención. Todos somos actores de estos procesos, tanto Ustedes, como yo, como otros/as. Nadie es ajeno a la comunicación, sin embargo, no todos utilizan procesos comunicativos en el contexto académico de una forma crítica y/o reflexiva. No olvide la máxima de Comunicación Efectiva: nunca utilizamos juicios valóricos, además dejamos las emociones de lado: éstos dos aspectos no son constitutivos de objetividad y solo aportan sesgo. Lo que he tenido oportunidad de observar, me han relatado o le han ocurrido a otros estudiantes.
\end{frame}

\begin{frame}[label={sec:orga3a838d}]{Consigna}
Identifique en estas oraciones/situaciones los actores en el mensaje, señalando quiénes manifiestan enfado, juicios valóricos, quiénes ejercen poder, quiénes son una amenaza, etc. Junto con ello, proponga una solución a estos actos comunicativos que se acerque a lo que hemos entendido por comunicación efectiva.
\end{frame}
\begin{frame}[label={sec:org9d81373}]{}
El estudiante no podrá venir a clases nunca los días sábado y lo comunica con un recado para el profesor.
\end{frame}
\begin{frame}[label={sec:org538dcc9}]{}
El profesor faltó a clases, no lo comunicó con anticipación, lo esperamos todos y nos fuimos enojados, porque el profesor es irresponsable y mala gente.
\end{frame}
\begin{frame}[label={sec:orgc0bf752}]{}
El profesor tomó una prueba, sólo para perjudicarnos.
\end{frame}
\begin{frame}[label={sec:orgec94c4e}]{}
En esta Universidad son todos chantas. Sólo para cobrar tienen tiempo.
\end{frame}
\begin{frame}[label={sec:org92d1d03}]{}
Nadie avisó que había clases en esta otra sala. No sé para qué publican horarios si los cambian y no los corrigen.
\end{frame}
\begin{frame}[label={sec:orge966fec}]{}
El profe puso en su face, que no va a venir hoy día.
\end{frame}
\begin{frame}[label={sec:orgbdf948d}]{}
El profe dejó pasar a un colega que sólo generó confusión en la sala y además se burló de nosotros.
\end{frame}
\begin{frame}[label={sec:orgb3991b4}]{}
Como yo soy adulto y profesional trato de una manera simétrica al profe, total tengo su misma edad, lo tuteo, lo increpo, le hago señalamientos y como no me gustó le dije lo que sentí.
\end{frame}
\begin{frame}[label={sec:org437b98c}]{}
El profe lee, lee y lee las presentaciones Power Point. Como si nosotros no supiéramos leer.
\end{frame}
\begin{frame}[label={sec:orgadd4965}]{}
Me leí la materia de la prueba anoche pero no entendí nada. ¿Quién me explica?
\end{frame}
\end{document}