% Created 2017-04-22 Sat 20:21
% Intended LaTeX compiler: pdflatex
%!TEX program = xelatex
\documentclass[12pt,letterpaper,article,x11names]{memoir}
\usepackage{pbox}
\usepackage[tight,%
            undotted,%
            nohints,%
            spanish]{minitoc}
\usepackage{hyperref}
\usepackage{abstract}
\usepackage{fourier}
\usepackage{wallpaper}
\usepackage{memoir-article-styles}
\usepackage{fontspec}
\usepackage{amssymb,amsmath}
\usepackage{polyglossia}
\setdefaultlanguage{spanish}
\usepackage[style=spanish]{csquotes}
\usepackage[hyperref=false,%
          backend=biber,%
          style=apa,%
          citetracker=true,%
          pagetracker=true]{biblatex}
\DeclareLanguageMapping{spanish}{spanish-apa}
\addbibresource{/Users/hsigrist/Dropbox/bibliography/references.bib}
\usepackage{polyglossia}
\setdefaultlanguage{spanish}
\usepackage[style=spanish]{csquotes}
\usepackage{pstricks-add}
\usepackage{tkz-euclide}
\usetkzobj{all}
\usepackage{pgf,tikz}
\usetikzlibrary{arrows}
\usepackage{siunitx}
\usepackage{xcolor}
\usepackage{booktabs}
\usepackage{marvosym}
\usepackage{longtable}
\usepackage[left=2cm,%
          right=2cm,%
          top=2cm,%
          bottom=3cm,%
          paperheight=33cm]{geometry}
\setlength{\absleftindent}{0pt}
\setlength{\absrightindent}{0pt}
\definecolor{links}{HTML}{000000}
\setromanfont[Mapping={tex-text}]{LinLibertineO}
\setsansfont[Mapping={tex-text}]{LinBiolinumO}
\setmonofont[Mapping={tex-text},Scale=0.8]{PragmataPro-Regular}
\graphicspath{{/Users/hsigrist/Dropbox/images/}}
\chapterstyle{article-2}
\pagestyle{kjh}
\definecolor{bluac}{RGB}{7,29,66}
\newtheorem{teorema}{Teorema}[chapter]
\newtheorem{lema}[teorema]{Lema}
\newtheorem{proposicion}[teorema]{Proposición}
\newtheorem{corolario}[teorema]{Corolario}
\newtheorem{definicion}[teorema]{Definición}
\newtheorem{ejemplo}[teorema]{Ejemplo}
\newtheorem{nota}[teorema]{Nota}
\title{}
      

\published{Ensayo no publicado. No citar sin permiso.}
\usepackage{tkz-fct}
\usepackage[usenames,dvipsnames]{xcolor}
\author{Hans Sigrist}
\date{Wed Apr 19 19:17:23 2017}
\title{Tercero Medio TP 2017\\\medskip
\large Ruta de Aprendizaje Matemática}
\hypersetup{
 pdfauthor={Hans Sigrist},
 pdftitle={Tercero Medio TP 2017},
 pdfkeywords={matemática, tercero medio, TP},
 pdfsubject={Recursos, estrategias, planificación de la asignatura.},
 pdfcreator={Emacs 25.1.1 (Org mode 9.0.5)}, 
 pdflang={Spanish}}
\begin{document}

\maketitle
\begin{abstract}
Recursos, estrategias, notas de clase y material complementario de la asignatura Matemática, dirigido a estudiantes de Tercer año Enseñanza Media en régimen Técnico Profesional (TP).
\end{abstract}

\setcounter{tocdepth}{5}
\tableofcontents

\chapter{Unidad 1 Números Complejos}
\label{sec:org403c076}
\section{Números imaginarios}
\label{sec:org71e1ccc}
\section{Potencias de números imaginarios}
\label{sec:org880b18d}
\section{Módulo y conjugado de un número complejo}
\label{sec:orge06094b}
\section{Operatoria combinada con complejos}
\label{sec:org111626a}
\subsection{Suma y resta de complejos}
\label{sec:org0957ee3}
\subsection{Multiplicación}
\label{sec:org914041a}
\subsection{División}
\label{sec:org05b0c08}
\chapter{Unidad 2 Ecuaciones de segundo grado}
\label{sec:orgb298c09}
\section{Syllabus}
\label{sec:orgb53cc35}
Entre las \textbf{Metas de Aprendizaje} de la actual unidad se encuentran:

\begin{itemize}
\item Modelar situaciones o fenómenos cuyos modelos resultantes sean funciones cuadráticas.
\item Comprender que toda ecuación de segundo grado con coeficientes reales tiene raíces en el conjunto de los números complejos.
\item Formular conjeturas, verificar para casos particulares y demostrar proposiciones utilizando conceptos, propiedades o relaciones de los diversos temas tratados en el nivel, y utilizar heurísticas para resolver problemas combinando, modificando o generalizando estrategias conocidas, fomentando la actitud reflexiva y crítica en la resolución de problemas.
\item Interesarse por conocer la realidad y utilizar el conocimiento.
\item Comprender y valorar la perseverancia, el rigor y el cumplimiento, la flexibilidad y la originalidad.
\end{itemize}

\begin{center}
\includegraphics[width=.9\linewidth]{esquema_unidad2.png}
\end{center}

\section{Ecuaciones cuadráticas}
\label{sec:orgd9d1b97}
\subsection{Panorama}
\label{sec:orgf71fbb2}
\href{mmental1-ecuacion-cuadratica.png}{\begin{center}
\includegraphics[width=.9\linewidth]{mmental1-ecuacion-cuadratica.png}
\end{center}}

\subsection{Recursos}
\label{sec:org62b490f}
\begin{enumerate}
\item \href{Guia\_Formativa\_2\_3TP\_Ecuacion\_Segundo\_Grado.pdf}{\begin{center}
\includegraphics[width=.9\linewidth]{Guia_Formativa_2_3TP_Ecuacion_Segundo_Grado.pdf}
\end{center}}. Este material provee una serie de 90 ejercicios para apoyar su aprendizaje en los \emph{métodos de resolución de ecuaciones de segundo grado} propios de la \textbf{Unidad 2 Ecuaciones de segundo grado}.
\item \href{Mapa\_Mental\_Unidad2\_Ecuaciones\_Cuadraticas.pdf}{\begin{center}
\includegraphics[width=.9\linewidth]{Mapa_Mental_Unidad2_Ecuaciones_Cuadraticas.pdf}
\end{center}}. Mapa Mental de la unidad ecuaciones cuadráticas, refiérase a esta ficha para apreciar un \textbf{panorama de las principales características de este tipo de ecuaciones}.
\end{enumerate}

\subsection{Problema de apertura}
\label{sec:org668258f}
Suponga que se desea cercar un terreno cuadrado cuya área conocida es de \(552.25m^2\), al respecto:

\begin{itemize}
\item ¿Cuál es el perímetro del terreno?
\item ¿Cómo obtiene la dimensión del lado del cuadrado?
\end{itemize}

\begin{center}
\includegraphics[width=.9\linewidth]{apertura.png}
\end{center}



Si ya reflexionó en torno a estas ideas, lo invito a observar el siguiente video que lo puede orientar más aún.

\begin{itemize}
\item ¿Por qué se desecha la solución \(x_2=-23.5\)?
\item ¿Por qué valor absoluto provee \textbf{dos soluciones}, una positiva y una negativa?
\end{itemize}

\subsection{Ecuaciones incompletas de la forma \(ax^2+c=0\)}
\label{sec:org85d63ab}
Estas ecuaciones se caracterizan por la ausencia del término lineal, debido a que el \textbf{coeficiente lineal}, \(b=0\).
En consecuencia, estas ecuaciones se resuelven mediante \emph{despeje} de la incógnita, a modo de ejemplo, observe la siguiente ecuación:
\begin{eqnarray*}
  4x^{2}-16&=&0\\
  4x^{2}&=&16\\
  x^{2}&=&4\\
  |x|&=&\sqrt{4}\\
  x&=&\pm2\\
   &\Rightarrow&x_1=2\wedge x_2=-2
\end{eqnarray*}

Lo anterior, permite establecer que toda ecuación de segundo grado de la forma \(ax^2+c=0\), se puede resolver mediante un procedimiento similar al del ejemplo.

Para profundizar, puede observar el siguiente video:

\subsection{Ecuaciones incompletas de la forma \(ax^2+bx=0\)}
\label{sec:orgc3a2384}

\subsection{Ecuaciones cuadráticas trinomio factorizable de la forma \(ax^2+bx+c=0\)}
\label{sec:org546daaa}
En este tipo de ecuaciones cuadráticas, están presentes todos los coeficientes, i.e., \(a,b,c\), particularmente el caso en que \(a=1\), de modo que es posible observar tan sólo los coeficientes lineal e independiente.

La estrategia consiste en \textbf{encontrar dos números que multiplicados den el valor de \(c\) y los mismos números sumados den el valor de \(b\)}.
\subsection{Ecuaciones completas mediante fórmula general \(ax^2+bx+c=0\)}
\label{sec:org6394794}

\subsection{Aplicaciones a problemas no rutinarios, complejos y no familiares (CUN)}
\label{sec:orgad6618d}

\section{Funciones cuadráticas}
\label{sec:org8f2f990}
\subsection{Representación de la función cuadrática mediante tablas y gráficos, y algebraicamente}
\label{sec:org89db61f}
\begin{enumerate}
\item Caso 1
\label{sec:org0013b07}


\begin{center}
\fbox{
\begin{minipage}[c]{.6\textwidth}
1. Dada la función \(f(x)=x^2+2x+1\), encuentre pares \((x,y)\) que cumplen con la igualdad y anótelos en la tabla siguiente:

\rule[.8em]{\textwidth}{2pt}

\begin{center}
\begin{tabular}{lrrrrrrr}
\(x\) & 0 & 1 & -1 & -2 & -3 & 2 & -5\\
\hline
\(y\) &  &  &  &  &  &  & \\
\end{tabular}
\end{center}
\end{minipage}
}
\end{center}

\begin{center}
\fbox{
\begin{minipage}[c]{.6\textwidth}
2. Represente estos pares ordenados en el plano cartesiano, busque otros puntos y verifique si pertenecen o no a la gráfica de la función.

\end{minipage}
}
\end{center}

\begin{center}
\fbox{
\begin{minipage}[c]{.6\textwidth}
3. Grafique la función en el plano cartesiano.

\end{minipage}
}
\end{center}

\begin{center}
\fbox{
\begin{minipage}[c]{.6\textwidth}
4. Analice el significado del par ordenado \((-1,0)\) y su relación el valor del discriminante igual a cero, en este caso particular.

\end{minipage}
}
\end{center}

\begin{tikzpicture}[scale=1.25]
     \tkzInit[xmin=-3,xmax=3,ymax=5]
     \tkzGrid
     \tkzAxeXY
     \tkzFct[color=red]{(x-1)*(x-1)+2}
   \end{tikzpicture}
\end{enumerate}
\end{document}